This package provides wrappers for using various persistent homology and other Topological Data Analysis software packages through python.

The code is an compilation of work done by \href{http://www.elizabethmunch.com/math}{\tt Elizabeth Munch} along with her students and collaborators. People who have contributed to teaspoon include\+:


\begin{DoxyItemize}
\item \href{http://www.firaskhasawneh.com}{\tt Firas Khasawneh}
\item Brian Bollen
\end{DoxyItemize}

\section*{Requirements }

Please note that this code is still pre-\/alpha, so many things are not fully up and running yet.

Bugs reports and feature reqests can be posted on the \href{https://gitlab.msu.edu/TSAwithTDA/teaspoon/issues}{\tt gitlab issues} page.

In order to use all wrappers in teaspoon.\+T\+DA, the following need to be installed so they can be run command line. Note\+: the teaspoon installation will not install or check for their install.


\begin{DoxyItemize}
\item \href{https://github.com/Ripser/ripser}{\tt Ripser}. Code by Ulrich Bauer for computing persistent homology of a point cloud or distance matrix.
\item \href{http://people.maths.ox.ac.uk/nanda/perseus/index.html}{\tt Perseus}. Code by Vidit Nanda for computing persistent homology of point clouds, cubical complexes, and distance matrices.
\item \href{https://bitbucket.org/grey_narn/hera}{\tt Hera}. Code by Michael Kerber, Dmitriy Morozov, and Arnur Nigmetov for computing bottleneck and Wasserstein distances.
\end{DoxyItemize}

\section*{Installation }

\subsubsection*{Warning\+:}

This code is still pre-\/alpha. In particular, using the pip install seems to be finicky at best. {\itshape T\+O\+DO\+: Need to find someone who knows more about pip install to fix this up.} {\bfseries If you are having installation issues, please make a note of what you\textquotesingle{}ve done, including copying error message outputs, as a comment into the \href{https://gitlab.msu.edu/TSAwithTDA/teaspoon/issues/1}{\tt installation issue on gitlab} so we can start figuring out what is up with this system.}

\subsubsection*{Installing using pip\+:}

As this code is still pre-\/alpha, your best bet is to cd into the folder containing teaspoon (should have setup.\+py there) and run


\begin{DoxyCode}
1 pip install -e .
\end{DoxyCode}


This is the developmental installion for pip. When things get more stable, we can remove the -\/e part.

\subsubsection*{Installing using python\+:}

According to the internet, the pip version of install appears to be better. However, if you don\textquotesingle{}t use pip, another option is to cd into the teaspoon directory and run\+:


\begin{DoxyCode}
1 python setup.py develop
\end{DoxyCode}


Again, this is the developer version of the installation. Eventually, we will want to be doing


\begin{DoxyCode}
1 python setup.py install
\end{DoxyCode}


\section*{Documentation }

Documentation is done using \href{http://www.doxygen.org}{\tt doxygen}. Documentation can be found in the \href{https://gitlab.msu.edu/TSAwithTDA/teaspoon/doc/html/index.html}{\tt doc folder}. Further info can be found in the https\+://gitlab.msu.\+edu/\+T\+S\+Awith\+T\+D\+A/teaspoon/blob/master/\+C\+O\+N\+T\+R\+I\+B\+U\+T\+I\+NG.md \char`\"{}contributing\char`\"{} page.

\section*{Contributing }

See the https\+://gitlab.msu.\+edu/\+T\+S\+Awith\+T\+D\+A/teaspoon/blob/master/\+C\+O\+N\+T\+R\+I\+B\+U\+T\+I\+NG.md \char`\"{}contributing\char`\"{} page for more information on workflows. 